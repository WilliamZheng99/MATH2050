\documentclass[12pt]{article}%
\usepackage{amsfonts}
\usepackage{fancyhdr}
\usepackage{comment}
\usepackage[a4paper, top=2.5cm, bottom=2.5cm, left=2.2cm, right=2.2cm]%
{geometry}
\usepackage{times}
\usepackage{amsmath}
\usepackage{changepage}
\usepackage{stfloats}
\usepackage{amssymb}
\usepackage{graphicx}
\usepackage{indentfirst}
\setlength{\parindent}{2em}
\setcounter{MaxMatrixCols}{30}
\newtheorem{theorem}{Theorem}
\newtheorem{acknowledgement}[theorem]{Acknowledgement}
\newtheorem{algorithm}[theorem]{Algorithm}
\newtheorem{axiom}{Axiom}
\newtheorem{case}[theorem]{Case}
\newtheorem{claim}[theorem]{Claim}
\newtheorem{conclusion}[theorem]{Conclusion}
\newtheorem{condition}[theorem]{Condition}
\newtheorem{conjecture}[theorem]{Conjecture}
\newtheorem{corollary}[theorem]{Corollary}
\newtheorem{criterion}[theorem]{Criterion}
\newtheorem{definition}[theorem]{Definition}
\newtheorem{example}[theorem]{Example}
\newtheorem{exercise}[theorem]{Exercise}
\newtheorem{lemma}[theorem]{Lemma}
\newtheorem{notation}[theorem]{Notation}
\newtheorem{problem}[theorem]{Problem}
\newtheorem{proposition}[theorem]{Proposition}
\newtheorem{remark}[theorem]{Remark}
\newtheorem{solution}[theorem]{Solution}
\newtheorem{summary}[theorem]{Summary}
\newenvironment{proof}[1][Proof]{\textbf{#1.} }{\ \rule{0.5em}{0.5em}}

\usepackage{mathtools}

\newcommand{\Q}{\mathbb{Q}}
\newcommand{\R}{\mathbb{R}}
\newcommand{\C}{\mathbb{C}}
\newcommand{\Z}{\mathbb{Z}}

\begin{document}

\title{MATH2050A Homework 7}
\author{ZHENG Weijia William, 1155124322}
\date{Spring, 2020}
\maketitle


\section{Q7 P134}
Let $f:[0,1] \to \mathbb{R}$ be $\forall x\in [0,1]$:

$$f(x)=\left\{
\begin{aligned}
1 &,~ x\in \mathbb{Q}\\
-1 &, ~x\not\in \mathbb{Q} \\
\end{aligned}
\right.$$

Note that $\forall x\in \mathbb{Q},$ we can find a sequence $(x_n)$ s.t. $\lim_{n \to \infty}x_n = x$ and $x_n \not\in \mathbb{Q}$, which implies $$\lim_{n \to \infty}f(x_n)=-1<f(x)=1.$$ 

Hence $\forall x\in \mathbb{Q},$ f is discontinuous. Also note that $\forall x\not\in \mathbb{Q},$ we can find a sequence $(x_n)$ s.t. $\lim_{n \to \infty}x_n = x$ and $x_n \in \mathbb{Q}$, which implies $$\lim_{n \to \infty}f(x_n)=1>f(x)=-1.$$ 

Therefore $\forall x\not\in \mathbb{Q},$ f is discontinuous.

Consider $|f|$, which is $|f(x)|=1, \forall x\in [0,1].$ So $|f|$ is constant on its domain and continuous. 


\section{Q15 P134}
According to the definition, $h(x)=\sup\{f(x),g(x)\}.$

$\forall x\in \mathbb{R},$ if $f(x)\leq g(x)$, then $h(x)=\sup\{f(x),g(x)\}=g(x).$

Note that $\frac{1}{2}(h(x)+g(x))+\frac{1}{2}|f(x)-g(x)|=\frac{1}{2}(f(x)+g(x))+\frac{1}{2}(g(x)-f(x))=g(x).$

Hence $h(x)=\frac{1}{2}(h(x)+g(x))+\frac{1}{2}|f(x)-g(x)|, \forall x \in \{x\in \mathbb{R}: f(x)\leq g(x)\}$

~

If $f(x)>g(x)$, then $h(x)=\sup\{f(x),g(x)\}=f(x).$ 
Note that $\frac{1}{2}(h(x)+g(x))+\frac{1}{2}|f(x)-g(x)|=\frac{1}{2}(f(x)+g(x))+\frac{1}{2}(f(x)-g(x))=f(x).$

Hence $h(x)=\frac{1}{2}(h(x)+g(x))+\frac{1}{2}|f(x)-g(x)|, \forall x \in \{x\in \mathbb{R}: f(x)>g(x)\}$

~

So, $h(x)=\frac{1}{2}(h(x)+g(x))+\frac{1}{2}|f(x)-g(x)|, \forall x \in \mathbb{R}.$ Then (i) is proved. 


~

Note that by the question, $f(x)$ and $g(x)$ are both continuous at $c$, hence $f(x)-g(x)$ is continuous at $c$. 

Therefore $|f(x)-g(x)|$ is continuous at $c$. Then $h(x)=\frac{1}{2}(h(x)+g(x))+\frac{1}{2}|f(x)-g(x)|, \forall x \in \mathbb{R}$ is continuous at $c$. So (ii) is proved. 




\end{document}
