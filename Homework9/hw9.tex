\documentclass[12pt]{article}%
\usepackage{amsfonts}
\usepackage{fancyhdr}
\usepackage{comment}
\usepackage[a4paper, top=2.5cm, bottom=2.5cm, left=2.2cm, right=2.2cm]%
{geometry}
\usepackage{times}
\usepackage{amsmath}
\usepackage{changepage}
\usepackage{stfloats}
\usepackage{amssymb}
\usepackage{graphicx}
\usepackage{indentfirst}
\setlength{\parindent}{2em}
\setcounter{MaxMatrixCols}{30}
\newtheorem{theorem}{Theorem}
\newtheorem{acknowledgement}[theorem]{Acknowledgement}
\newtheorem{algorithm}[theorem]{Algorithm}
\newtheorem{axiom}{Axiom}
\newtheorem{case}[theorem]{Case}
\newtheorem{claim}[theorem]{Claim}
\newtheorem{conclusion}[theorem]{Conclusion}
\newtheorem{condition}[theorem]{Condition}
\newtheorem{conjecture}[theorem]{Conjecture}
\newtheorem{corollary}[theorem]{Corollary}
\newtheorem{criterion}[theorem]{Criterion}
\newtheorem{definition}[theorem]{Definition}
\newtheorem{example}[theorem]{Example}
\newtheorem{exercise}[theorem]{Exercise}
\newtheorem{lemma}[theorem]{Lemma}
\newtheorem{notation}[theorem]{Notation}
\newtheorem{problem}[theorem]{Problem}
\newtheorem{proposition}[theorem]{Proposition}
\newtheorem{remark}[theorem]{Remark}
\newtheorem{solution}[theorem]{Solution}
\newtheorem{summary}[theorem]{Summary}
\newenvironment{proof}[1][Proof]{\textbf{#1.} }{\ \rule{0.5em}{0.5em}}

\usepackage{mathtools}

\newcommand{\Q}{\mathbb{Q}}
\newcommand{\R}{\mathbb{R}}
\newcommand{\C}{\mathbb{C}}
\newcommand{\Z}{\mathbb{Z}}

\begin{document}

\title{MATH2050A Homework 9}
\author{William Zheng}
\date{Spring, 2020}
\maketitle


\section{Q8 P148}
Note that $f(x)$ is uniformly continuous on $\mathbb{R}$, i.e., $\forall \epsilon_1>0,$ 

$\exists \delta_1>0$ s.t. $\forall x_1,x_2\in \mathbb{R}, |x_1-x_2|<\delta_1,$ we have $$|f(x_1)-f(x_2)|<\epsilon_1.$$ 

And since $g(x)$ is also uniformly continuous on $\mathbb{R}$, $\forall \epsilon_2>0, \exists \delta_2>0$ s.t. $\forall x_1,x_2\in \mathbb{R}, |x_1-x_2|<\delta_2,$ we have $$|g(x_1)-g(x_2)|<\epsilon_2.$$ 

Take $\epsilon_2 = \delta_1>0.$ We have some $\delta_2>0$ s.t. $\forall x_1,x_2\in \mathbb{R}, |x_1-x_2|<\delta_2,$ we have $$|g(x_1)-g(x_2)|<\delta_1.$$ 

Note that $g(x_1),g(x_2) \in \mathbb{R}.$ Then by the previous part we wrote at the beginning of the proof, $$ |f(g(x_1))-f(g(x_2))|<\epsilon_1.$$

Hence, by the definition of uniformly continuous, $f(g(x))$ is uniformly continuous on $\mathbb{R}$.


~\

\section{Q10 P148}

Note that $f(x)$ is uniformly continuous, then $\forall \epsilon>0, \exists \delta>0$ s.t. $\forall x_1,x_2\in A, |x_1-x_2|<\delta,$ we have $$|f(x_1)-f(x_2)|<\epsilon.$$

Let $\epsilon=1$ and then fix a corresponding $\delta.$

Note that A is bounded. Hence A can be covered by a union of multiple open intervals, i.e., $\exists X_1,x_2,...,x_n \in A, B_1=(x_1-\delta, x_1+\delta), B_2=(x_2-\delta, x_2+\delta),...B_n=(x_n-\delta, x_n+\delta)$ s.t. $$A \subset B_1 \cup B_2 \cup ...\cup B_n.$$

$\forall x\in A$, $\exists m$ s.t. $x \in B_m.$ Because $B_m=(x_m-\delta, x_m+\delta)$, then $|x-x_m|<\delta$, which implies $$|f(x)-f(x_m)|<1.$$

Then $f(x)<f(x_m)+1$, $f(x)>f(x_m)-1$, which implies $\forall x\in A,$ $$f(x)<\sup\{f(x_1)+1,...f(x_n)+1\}=U$$ and $$f(x)>\inf\{f(x_1)-1,...f(x_n)-1\}=L.$$

Hence, $f$ is bounded on A. 


\end{document}
