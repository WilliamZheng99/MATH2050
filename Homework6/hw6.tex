\documentclass[12pt]{article}%
\usepackage{amsfonts}
\usepackage{fancyhdr}
\usepackage{comment}
\usepackage[a4paper, top=2.5cm, bottom=2.5cm, left=2.2cm, right=2.2cm]%
{geometry}
\usepackage{times}
\usepackage{amsmath}
\usepackage{changepage}
\usepackage{stfloats}
\usepackage{amssymb}
\usepackage{graphicx}
\usepackage{indentfirst}
\setlength{\parindent}{2em}
\setcounter{MaxMatrixCols}{30}
\newtheorem{theorem}{Theorem}
\newtheorem{acknowledgement}[theorem]{Acknowledgement}
\newtheorem{algorithm}[theorem]{Algorithm}
\newtheorem{axiom}{Axiom}
\newtheorem{case}[theorem]{Case}
\newtheorem{claim}[theorem]{Claim}
\newtheorem{conclusion}[theorem]{Conclusion}
\newtheorem{condition}[theorem]{Condition}
\newtheorem{conjecture}[theorem]{Conjecture}
\newtheorem{corollary}[theorem]{Corollary}
\newtheorem{criterion}[theorem]{Criterion}
\newtheorem{definition}[theorem]{Definition}
\newtheorem{example}[theorem]{Example}
\newtheorem{exercise}[theorem]{Exercise}
\newtheorem{lemma}[theorem]{Lemma}
\newtheorem{notation}[theorem]{Notation}
\newtheorem{problem}[theorem]{Problem}
\newtheorem{proposition}[theorem]{Proposition}
\newtheorem{remark}[theorem]{Remark}
\newtheorem{solution}[theorem]{Solution}
\newtheorem{summary}[theorem]{Summary}
\newenvironment{proof}[1][Proof]{\textbf{#1.} }{\ \rule{0.5em}{0.5em}}

\usepackage{mathtools}

\newcommand{\Q}{\mathbb{Q}}
\newcommand{\R}{\mathbb{R}}
\newcommand{\C}{\mathbb{C}}
\newcommand{\Z}{\mathbb{Z}}

\begin{document}

\title{MATH2050A Homework 6}
\author{ZHENG Weijia William, 1155124322}
\date{Spring, 2020}
\maketitle

\section{Q10 (Section 5.1)}
Let $\epsilon>0$. $\forall x_0 \in \mathbb{R},$

If $x_0>0$, take $\delta_1 = \inf{\{\frac{x_0}{2},\epsilon\}}.$ We have $$|f(x)-f(x_0)| = |x-x_0|<\epsilon,~~ \forall x:|x-x_0|<\delta_1.$$

If $x_0<0$, take $\delta_2 = \inf{ \{ \epsilon, \frac{-x_0}{2}  \} }$. We have $$|f(x)-f(x_0)| = ||x|-|x_0||=|-x+x_0|<\epsilon, ~~\forall x: |x-x_0|<\delta_2.$$

If $x_0=0$, take $\delta_3 = \epsilon.$ We have $$|f(x)-f(x_0)|=|f(x)|=|x|<\epsilon, ~~\forall x:|x-x_0|<\delta_3.$$

Base on all above, we have $\forall x_0 \in \mathbb{R}, \forall \epsilon>0,$ take $\delta=\inf{\{\delta_1,\delta_2, \delta_3\}}$, we have $$|f(x)-f(x_0)|<\epsilon,~~\forall x: |x-x_0|<\delta.$$

Hence, the absolute value function is continuous at every point $x_0 \in \mathbb{R}.$
~\

\section{Q4a (Section 5.1)}
Let $x_0\in \mathbb{R}.$ 

\subsection{Case 1: $x_0 \in \mathbb{Z}$}

If $x_0\in \mathbb{Z}.$ Then $f(x_0)=x_0$ itself. Take $\epsilon=0.5.$ $\forall \delta>0$, take $x = x_0+\frac{\delta}{2}$ we have $x > x_0, x_0\in \mathbb{Z}$, hence $f(x) \geq x_0+1 > f(x_0)=x_0.$ Therefore, we have $$f(x)-f(x_0)\geq 1, |f(x)-f(x_0)|\geq 1>\epsilon.$$

\subsection{Case 2: $x_0 \not\in \mathbb{Z}$, $x_0>0$}

If $x_0>0$, we know that by Archimedean's Property, $\exists n \in \mathbb{N}$ s.t. $$0<x_0<n.$$

Define $N=\inf\{n \in \mathbb{N}: x_0<n\}.$ As the set is a subset of natural number, by the well-ordering principle, the definition is valid. 

Note that $x_0>N-1.$ (Suppose not, $N-1$ wil be the inf of $\inf\{n \in \mathbb{N}: x_0<n\}$) Hence $N-1<x_0<N.$ 

Let $\epsilon >0.$ Take $\delta = \inf\{\frac{|N-1-x_0|}{2},\frac{|N-x_0|}{2}\}$, $\forall x:|x-x_0|<\delta,$ which implies $N-1<x<N$, we have $$|f(x)-f(x_0)|=|N-1-(N-1)|=0<\epsilon.$$ 

Therefore $f(x)$ is continuous on $x_0>0$ with $x_0 \in \mathbb{R}$ and $x_0 \not\in \mathbb{Z}$

\subsection{Case 3: $x_0 \not\in \mathbb{Z}$, $x_0<0$}
Consider $-x_0$, then -$x_0>0$ and we can apply the argument of case 2. So $\exists N \in \mathbb{N}$ s.t. $$N-1<-x_0<N.$$ which implies $$-N<x_0<-N+1.$$ Then $\forall \epsilon>0$, $\forall x:|x-x_0|<\delta$, where $\delta=\inf\{\frac{|-N-x_0|}{2},\frac{|-N+1-x_0|}{2}\}$ $$|f(x)-f(x_0)|=|-N-(-N)|=0<\epsilon.$$

~\

Hence the point of continuity of $f(x)$ is $\{ x \in \mathbb{R}: x \not\in \mathbb{Z}\}.$

\section{Q9 (Section 4.3)}
Let $g(x)=xf(x).$ So we have $\lim_{x \to \infty}g(x)=L.$ Define $h:(0,\infty) \to \mathbb{R}$ with $h(x)=\frac{1}{x}$. We have $\lim_{x \to \infty}h(x)=0.$ 

Note that $f(x)=g(x)\cdot h(x), \forall x\in (0,\infty).$ Hence $$\lim_{x \to \infty}f(x)=L\cdot 0=0.$$


\section{Q13 (Section 4.3)}
Let $\epsilon>0.$

By $\lim_{x \to \infty}f(x)=L.$ $\exists N$ s.t. $\forall x>N, |f(x)-L|<\epsilon.$

And by $\lim_{x \to \infty}g(x)=\infty.$ There exists such $K$ s.t. $\forall x>K$,$$ g(x)>N.$$

Therefore $\forall \epsilon>0,$ there exists $K$ such that $\forall x>K$, hence $g(x)>N$, we have $$|f(g(x))-L|<\epsilon.$$

Which implies $\lim_{x \to \infty}f\circ g=L.$




\end{document}
