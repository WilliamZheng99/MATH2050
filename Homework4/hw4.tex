\documentclass[12pt]{article}%
\usepackage{amsfonts}
\usepackage{fancyhdr}
\usepackage{comment}
\usepackage[a4paper, top=2.5cm, bottom=2.5cm, left=2.2cm, right=2.2cm]%
{geometry}
\usepackage{times}
\usepackage{amsmath}
\usepackage{changepage}
\usepackage{stfloats}
\usepackage{amssymb}
\usepackage{graphicx}
\usepackage{indentfirst}
\setlength{\parindent}{2em}
\setcounter{MaxMatrixCols}{30}
\newtheorem{theorem}{Theorem}
\newtheorem{acknowledgement}[theorem]{Acknowledgement}
\newtheorem{algorithm}[theorem]{Algorithm}
\newtheorem{axiom}{Axiom}
\newtheorem{case}[theorem]{Case}
\newtheorem{claim}[theorem]{Claim}
\newtheorem{conclusion}[theorem]{Conclusion}
\newtheorem{condition}[theorem]{Condition}
\newtheorem{conjecture}[theorem]{Conjecture}
\newtheorem{corollary}[theorem]{Corollary}
\newtheorem{criterion}[theorem]{Criterion}
\newtheorem{definition}[theorem]{Definition}
\newtheorem{example}[theorem]{Example}
\newtheorem{exercise}[theorem]{Exercise}
\newtheorem{lemma}[theorem]{Lemma}
\newtheorem{notation}[theorem]{Notation}
\newtheorem{problem}[theorem]{Problem}
\newtheorem{proposition}[theorem]{Proposition}
\newtheorem{remark}[theorem]{Remark}
\newtheorem{solution}[theorem]{Solution}
\newtheorem{summary}[theorem]{Summary}
\newenvironment{proof}[1][Proof]{\textbf{#1.} }{\ \rule{0.5em}{0.5em}}

\usepackage{mathtools}

\newcommand{\Q}{\mathbb{Q}}
\newcommand{\R}{\mathbb{R}}
\newcommand{\C}{\mathbb{C}}
\newcommand{\Z}{\mathbb{Z}}

\begin{document}

\title{MATH2050A HW4}
\author{ZHENG Weijia William, 1155124322}
\date{Spring, 2020}
\maketitle

\section{P84 Q4a}

The sequence in the question is $(x_n=1-(-1)^{n}+\frac{1}{n}).$

Suppose it is convergent, then $\exists L \in \mathbb{R}~s.t.$ $$\forall \epsilon>0, \exists N\in \mathbb{N^*}~s.t.~\forall n>N, |x_n-L|<\frac{\epsilon}{2}.$$

Which implies that $$\forall n,m>N, |x_n-x_m|\leq |x_n-L|+|L-x_m|< \epsilon.$$

Let $\epsilon=0.5,n=2N+2,m=2N+1.$ Note that $n,m>N.$ So we should have $$|x_n-x_m|<0.5$$. 

However, $$|x_n-x_m|=|1-(-1)^{2N+2}+\frac{1}{2N+2}-1+(-1)^{2N+1}-\frac{1}{2N+1}|=2+\frac{1}{2N+1}-\frac{1}{2N+2}>2.$$ 

This is contradict with the previous $|x_n-x_m|<0.5.$

Hence, the $(x_n)$ is divergent.


~\

\section{P84 Q7a}

We consider the sequence $a_k=(1+\frac{1}{k})^k$ on $\mathbb{R}.$ 

Let $k_n=n^2$, note that $a_{k_n}=(1+\frac{1}{n^2})^{n^2}$ is a subsequence of $(a_k)$.

~\

First we prove that $(a_k)$ is increasing. Note that $a_k>0, \forall k.$

So we inspect $$\frac{a_{k+1}}{a_k}=(\frac{k+2}{k+1})^k (\frac{k}{k+1})^k (\frac{k+2}{k+1})=(1+\frac{1}{1+k})[1-\frac{1}{(k+1)^2}]^k.$$

Note that $\frac{-1}{(k+1)^2}\geq -1$ and $k\geq 1$, so by Bernoulli's inequality, we have $$[1-\frac{1}{(k+1)^2}]^k\geq 1-\frac{k}{(k+1)^2}.$$

We have $$\frac{a_{k+1}}{a_k}\geq \frac{k+2}{k+1}(1-\frac{k}{(k+1)^2})=1+\frac{1}{(k+1)^3}\geq 1.$$

~\

Hence, we have already proved $(a_k)$ is increasing, then we'd prove $(a_k)$ is bounded. 

Consider $n\geq 2.$ Using binomial theorem, we have $$a_k=2+\sum_{i=2}^{k} \binom{k}{i}\frac{1}{k^i}=2+\sum_{i=2}^{k} \frac{k(k-1)...(k-i+1)}{k^i}\frac{1}{i!}\leq 2+ \sum_{i=2}^{k}\frac{1}{i!}\leq 2+\sum_{i=2}^{k}\frac{1}{2^{i-1}}=3-\frac{1}{2^{k-1}}.$$

Hence $a_k<3$, so $(a_k)$ is bounded.

So $(a_k)$ is bounded and increasing, hence $(a_k)$ is convergent. Which implies its sub-sequence $a_{k_n}$ converges and converges to the same limit as $(a_k)$ does. 

As $\lim_{k \to \infty} (1+\frac{1}{k})^k$ has limit and is defined to be $e$, so $$\lim_{n \to \infty} (1+\frac{1}{n^2})^{n^2}=e.$$

~\

\section{P84 Q8a}

Claim $\lim_{n \to \infty} n^{\frac{1}{n}}=0.$ 

Consider $n\geq 2.$ By the definition of limit, $\forall \epsilon >0$ (we further consider $\epsilon <1.$) 
$$\exists N \in \mathbb{N}~s.t.~|n^{\frac{1}{n}}-1|<\epsilon, \forall n>N.$$

Which is equivalent to $(1-\epsilon)^n<n<(1+\epsilon)^n.$ 

The inequality in the left hand side holds naturally because $(1-\epsilon)^n<1<n.$

~\

Hence we need to find such $N$ s.t. $\forall n>N$ $$n<(1+\epsilon)^n.$$

Using binomial theorem, we have $$(1+\epsilon)^n>\binom{n}{2}\epsilon^2=\frac{n(n-1)}{2}\epsilon^2.$$

Suffice to let $$\frac{n(n-1)}{2}\epsilon^2>n.$$ 

Which is equivalent to $n>1+\frac{2}{\epsilon^2}.$ So it suffices to let $N>1+\frac{2}{\epsilon^2}.$ 

Hence $$\forall n>N, |n^{\frac{1}{n}}-1|<\epsilon.$$ 

So we have $\lim_{n \to \infty}n^{\frac{1}{n}}=1.$

Hence $$\lim_{n \to \infty}(3n)^{\frac{1}{2n}}=\lim_{n \to \infty}((3n)^{\frac{1}{3n}})^{\frac{3}{2}}=1^{\frac{3}{2}}=1.$$

So the limit in this question is 1.

\end{document}
