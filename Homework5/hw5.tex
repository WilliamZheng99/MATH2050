\documentclass[12pt]{article}%
\usepackage{amsfonts}
\usepackage{fancyhdr}
\usepackage{comment}
\usepackage[a4paper, top=2.5cm, bottom=2.5cm, left=2.2cm, right=2.2cm]%
{geometry}
\usepackage{times}
\usepackage{amsmath}
\usepackage{changepage}
\usepackage{stfloats}
\usepackage{amssymb}
\usepackage{graphicx}
\usepackage{indentfirst}
\setlength{\parindent}{2em}
\setcounter{MaxMatrixCols}{30}
\newtheorem{theorem}{Theorem}
\newtheorem{acknowledgement}[theorem]{Acknowledgement}
\newtheorem{algorithm}[theorem]{Algorithm}
\newtheorem{axiom}{Axiom}
\newtheorem{case}[theorem]{Case}
\newtheorem{claim}[theorem]{Claim}
\newtheorem{conclusion}[theorem]{Conclusion}
\newtheorem{condition}[theorem]{Condition}
\newtheorem{conjecture}[theorem]{Conjecture}
\newtheorem{corollary}[theorem]{Corollary}
\newtheorem{criterion}[theorem]{Criterion}
\newtheorem{definition}[theorem]{Definition}
\newtheorem{example}[theorem]{Example}
\newtheorem{exercise}[theorem]{Exercise}
\newtheorem{lemma}[theorem]{Lemma}
\newtheorem{notation}[theorem]{Notation}
\newtheorem{problem}[theorem]{Problem}
\newtheorem{proposition}[theorem]{Proposition}
\newtheorem{remark}[theorem]{Remark}
\newtheorem{solution}[theorem]{Solution}
\newtheorem{summary}[theorem]{Summary}
\newenvironment{proof}[1][Proof]{\textbf{#1.} }{\ \rule{0.5em}{0.5em}}

\usepackage{mathtools}

\newcommand{\Q}{\mathbb{Q}}
\newcommand{\R}{\mathbb{R}}
\newcommand{\C}{\mathbb{C}}
\newcommand{\Z}{\mathbb{Z}}

\begin{document}

\title{MATH2050A Homework 5}
\author{ZHENG Weijia William, 1155124322}
\date{Spring, 2020}
\maketitle

\section{Q3(b) (Section 3.5)}
Let $x_n = n+\frac{(-1)^n}{n}.$ We want to show $(x_n)$ is not a Cauchy sequence.

Take $\epsilon=1$, $\forall N\in \mathbb{N}$, choose $m = 2N+1, n=2N+2.$ We have $$|x_m-x_n|=|2N+1+\frac{-1}{2N+1}-(2N+2)-\frac{-1}{2N+2}|=1+\frac{1}{(2N+1)(2N+2)}>1=\epsilon.$$

$\therefore$ $(x_n)$ is not a Cauchy sequence. 

~\

\section{Q5 (Section 3.5)}
Note that $$|x_{n+1}-x_n|=\sqrt{n+1}-\sqrt{n}=\frac{1}{\sqrt{n+1}+\sqrt{n}}.$$

$\forall \epsilon>0,$ $\exists N>\frac{1}{\epsilon^2}$ s.t. $\forall n>N$ we have $$|x_{n+1}-x_n-0|=\frac{1}{\sqrt{n+1}+\sqrt{n}}\leq \frac{1}{\sqrt{n}}<\epsilon.$$

$\therefore$ $\lim |x_{n+1}-x_n|=0.$

Then we would show that $(x_n)$ is unbounded. $\forall M>0, M\in \mathbb{N},$ choose $u=M^2+1$, we have $$x_u = \sqrt{M^2+1}>M.$$

So $(x_n)$ is unbounded. Also note that $(x_n)$ is increasing, so $(x_n)$ is divergent and hence not a Cauchy sequence.


\section{Q9 (Section 3.5)}
Let $m,n \in \mathbb{N}, m>n$, then we have $$|x_m-x_n|\leq|x_m-x_{m-1}|+...+|x_{n+1}-x_n|<r^{m-1}+...r^{n}=\frac{r^n-r^m}{1-r}\leq \frac{r^n}{1-r}.$$

$\forall \epsilon>0, \exists N>\frac{\ln{\epsilon(1-r)}}{\ln{r}}$ s.t. $\forall n>N,$ $$|x_m-x_n|<\epsilon.$$

$\therefore$ $(x_n)$ is a Cauchy sequence.
\end{document}
